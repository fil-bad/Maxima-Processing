%& -shell-escape
\documentclass[a4paper,12pt]{article}
\usepackage[legalpaper, portrait , margin=1.5cm]{geometry}
\usepackage[utf8]{inputenc}
\usepackage[T1]{fontenc}
\usepackage[italian]{babel}
\usepackage{wrapfig}
\usepackage{graphicx}
\graphicspath{ {./Sorgenti/} }
\usepackage{enumitem}
\usepackage{pifont}
\usepackage{amsmath}


\author{Alfano Emanuele \\ Badalamenti Filippo \\ Vitti Gabriele}

\begin{document}
\title{Cinematiche inverse complete}
\maketitle

In questa relazione andremo a vedere come si calcolano le cinematiche
  inverse complete di due robot attraverso il disaccoppiamento polso-strutura portante in cui, se sono soddisfatti certi vincoli algebrici e geometrici, allora il problema completo della cinematica inversa può essere ricondotto ai due sotto-problemi di cinematica inversa di posizione e di orientamento separati:

\begin{itemize}
\item Manipolatore di Stanford + Polso Sferico
\item Antropomorfo + Polso Sferico (Puma)
\end{itemize}

Risolvere un problema di cinematica inversa completa vuol dire che, assegnata una matrice di trasformazione del tipo:

\begin{center}
$\begin{pmatrix}
{r_{11}} & {r_{12}} & {r_{13}} & {{\mathit{df}}_1}\\
{r_{21}} & {r_{22}} & {r_{23}}&{{\mathit{df}}_2}\\
{r_{31}} & {r_{32}} & {r_{33}} & {{\mathit{df}}_3}\\
0 & 0 & 0 & 1
\end{pmatrix}$
\end{center}

Sapendo che:

\begin{center}
$d_f=\begin{pmatrix}{{\mathit{df}}_1}\\
{{\mathit{df}}_2}\\
{{\mathit{df}}_3}\end{pmatrix}$ e $R_f=\begin{pmatrix}{r_{11}} & {r_{12}} & {r_{13}}\\
{r_{21}} & {r_{22}} & {r_{23}}\\
{r_{31}} & {r_{32}} & {r_{33}}\end{pmatrix}$
\end{center}

dove $d_f$ è il vettore posizione del punto terminale del robot completo di Stanford (manipolatore + polso sferico), ricavato con la cinematica diretta del robot, rispetto al sistema di riferimento inerziale e $R_f$ è la matrice di rotazione che indica l'orientamento del punto terminale del robot completo di Stanford rispetto al sistema di riferimento inerziale.\\
 Bisogna determinare due vettori (almeno) $q_a$ e $q_b$ tali che: \\

		$\left\{ \begin{array} {ll} 
		R_f  = R_{03}(q_a) \cdot  R_{36}(q_b) \\
		d_f = R_{03}(q_a) \cdot d_{36}(q_b) + d_{03}(q_a)
		\end{array}
		\right.$ \\

dove $q_a$ è l'insieme delle variabili di giunto $q_1, q_2, q_3$ che servono per posizionare il punto terminale e $q_b$ è l'insieme delle variabili di giunto $q_4, q_5, q_6$ che servono per orientare il punto terminale.

\pagebreak

\section{Calcolo costanti attraverso la matrice di trasformazione del polso sferico}
Per prima cosa è necessario trovare i due termini costanti ($d_0$ e $d_1$) che permettono il disaccoppiamento polso-struttura portante. Data la matrice di trasformazione del polso sferico:

\begin{center}
$Q_{36}$ =$\begin{pmatrix}
{c_{{q_4}}} {c_{{q_5}}} {c_{{q_6}}}-{s_{{q_4}}} {s_{{q_6}}} & -{c_{{q_4}}} {c_{{q_5}}} {s_{{q_6}}}-{s_{{q_4}}} {c_{{q_6}}} & {c_{{q_4}}} {s_{{q_5}}} & {L_6} {c_{{q_4}}} {s_{{q_5}}}\\
{c_{{q_4}}} {s_{{q_6}}}+{s_{{q_4}}} {c_{{q_5}}} {c_{{q_6}}} & {c_{{q_4}}} {c_{{q_6}}}-{s_{{q_4}}} {c_{{q_5}}} {s_{{q_6}}} & {s_{{q_4}}} {s_{{q_5}}} & {L_6} {s_{{q_4}}} {s_{{q_5}}}\\
-{s_{{q_5}}} {c_{{q_6}}} & {s_{{q_5}}} {s_{{q_6}}} & {c_{{q_5}}} & {L_6} {c_{{q_5}}}\\
0 & 0 & 0 & 1
\end{pmatrix}$
\end{center}
Abbiamo che:

\begin{center}
$R_{36}$ = $\begin{pmatrix}{c_{{q_4}}} {c_{{q_5}}} {c_{{q_6}}}-{s_{{q_4}}} {s_{{q_6}}} & -{c_{{q_4}}} {c_{{q_5}}} {s_{{q_6}}}-{s_{{q_4}}} {c_{{q_6}}} & {c_{{q_4}}} {s_{{q_5}}}\\
{c_{{q_4}}} {s_{{q_6}}}+{s_{{q_4}}} {c_{{q_5}}} {c_{{q_6}}} & {c_{{q_4}}} {c_{{q_6}}}-{s_{{q_4}}} {c_{{q_5}}} {s_{{q_6}}} & {s_{{q_4}}} {s_{{q_5}}}\\
-{s_{{q_5}}} {c_{{q_6}}} & {s_{{q_5}}} {s_{{q_6}}} & {c_{{q_5}}}\end{pmatrix}$
\space\space\space\space\space   
$d_{36}$ = $\begin{pmatrix}{L_6} {c_{{q_4}}} {s_{{q_5}}}\\
{L_6} {s_{{q_4}}} {s_{{q_5}}}\\
{L_6} {c_{{q_5}}}\end{pmatrix}$
\end{center}

Da queste due matrici cerchiamo di ricavare le due costanti ($d_0$ e $d_1$) che rendono vera la seguente equazione:

$$ d_{36} = R_{36} \cdot d_1 + d_0 $$

Trovando $d_0$ e $d_1$ sarà possibile disaccoppiare la soluzione delle due cinematiche inverse di posizione e di orientamento.





\begin{center}
$\begin{pmatrix}{L_6} {c_{{q_4}}} {s_{{q_5}}}\\
{L_6} {s_{{q_4}}} {s_{{q_5}}}\\
{L_6} {c_{{q_5}}}\end{pmatrix}$
=
$\begin{pmatrix}{c_{{q_4}}} {c_{{q_5}}} {c_{{q_6}}}-{s_{{q_4}}} {s_{{q_6}}} & -{c_{{q_4}}} {c_{{q_5}}} {s_{{q_6}}}-{s_{{q_4}}} {c_{{q_6}}} & {c_{{q_4}}} {s_{{q_5}}}\\
{c_{{q_4}}} {s_{{q_6}}}+{s_{{q_4}}} {c_{{q_5}}} {c_{{q_6}}} & {c_{{q_4}}} {c_{{q_6}}}-{s_{{q_4}}} {c_{{q_5}}} {s_{{q_6}}} & {s_{{q_4}}} {s_{{q_5}}}\\
-{s_{{q_5}}} {c_{{q_6}}} & {s_{{q_5}}} {s_{{q_6}}} & {c_{{q_5}}}\end{pmatrix}$
$\cdot$
$\begin{pmatrix}\mathit{a1}\\
\mathit{b1}\\
\mathit{c1}\end{pmatrix}$
+
$\begin{pmatrix}\mathit{a0}\\
\mathit{b0}\\
\mathit{c0}\end{pmatrix}$
\end{center}

Dove:
\begin{center}
$ d_1 $=$\begin{pmatrix}\mathit{a1}\\
\mathit{b1}\\
\mathit{c1}\end{pmatrix}$
\space
$ d_0 $=$\begin{pmatrix}\mathit{a0}\\
\mathit{b0}\\
\mathit{c0}\end{pmatrix}$
\end{center}


Risolvendo sia con $d_0$ che con $d_1$ si ottengono $\infty$ soluzioni, ma si vede anche a occhio che una soluzione semplice è:

\begin{center}
$ d_1 $=$\begin{pmatrix}
0\\
0\\
L_6\end{pmatrix}$
\space
$ d_0 $=$\begin{pmatrix}
0\\
0\\
0\end{pmatrix}$
\end{center}

\section{Orientamento inverso del Polso Sferico}
Nel calcolo disaccoppiato, in seguito al calcolo di $R_{03}(q_a)$ con la cinematica inversa di pura posizione (come fatto nel capitolo 3.1: Calcolo posizione disaccoppiata), sono in grado di ricavare la matrice di orientamento del polso sferico $R_{36}(q_b)$.
Portandoci avanti, siamo già in grado di ricavare $R_{36}(q_b)$ e le variabili $q_4,q_5,q_6$ (le quali servono per determinare l'orientamento del polso sferico) attraverso una cinematica inversa di puro orientamento. Useremo i risultati, che ora ricaveremo, nel capitolo 3.2: Calcolo orientamento disaccoppiato.


Nel calcolo disaccoppiato risulta che l'orientamento del polso sferico è pari a:

\begin{center}
$ R_{36}(q_b) =  R_{03}^T(q_a) \cdot R_f $
 \end{center}


Essendo però la matrice di orientamento del polso sferico $R_{36}(q_b)$ esattamente
pari alla matrice di orientamento della terna di Eulero $ R_{zyz} $,
trovare $ \alpha,\beta,\gamma $ della matrice $ R_{zyz}(\alpha,\beta,\gamma) $ equivale a trovare $q_4,q_5,q_6$ della matrice $R_{36}(q_b)$:

\begin{center}
$ R_{zyz}(\alpha,\beta,\gamma)$ = $\begin{pmatrix}\operatorname{c}\left( \alpha \right)  \operatorname{c}\left( \beta \right)  \operatorname{c}\left( \gamma \right) -\operatorname{s}\left( \alpha \right)  \operatorname{s}\left( \gamma \right)  & -\operatorname{c}\left( \alpha \right)  \operatorname{s}\left( \gamma \right) -\operatorname{s}\left( \alpha \right)  \operatorname{c}\left( \beta \right)  \operatorname{c}\left( \gamma \right)  & \operatorname{s}\left( \beta \right)  \operatorname{c}\left( \gamma \right) \\
\operatorname{c}\left( \alpha \right)  \operatorname{c}\left( \beta \right)  \operatorname{s}\left( \gamma \right) +\operatorname{s}\left( \alpha \right)  \operatorname{c}\left( \gamma \right)  & \operatorname{c}\left( \alpha \right)  \operatorname{c}\left( \gamma \right) -\operatorname{s}\left( \alpha \right)  \operatorname{c}\left( \beta \right)  \operatorname{s}\left( \gamma \right)  & \operatorname{s}\left( \beta \right)  \operatorname{s}\left( \gamma \right) \\
-\operatorname{c}\left( \alpha \right)  \operatorname{s}\left( \beta \right)  & \operatorname{s}\left( \alpha \right)  \operatorname{s}\left( \beta \right)  & \operatorname{c}\left( \beta \right) \end{pmatrix}$
\end{center}


Definito: $R_{03}^T(q_a) \cdot R$ = $\begin{pmatrix}{r_{11}} & {r_{12}} & {r_{13}}\\
{r_{21}} & {r_{22}} & {r_{23}}\\
{r_{31}} & {r_{32}} & {r_{33}}\end{pmatrix}$.
Avremo che $q_4,q_5,q_6$ sono:

\begin{itemize}
\item  $c(\beta) = r_{33} \Longrightarrow s(\beta) = \pm \sqrt{1-r_{33}^2}
		\hfill \longrightarrow \hfill
		\beta = atan2 (\pm s(\beta), c(\beta)) = \left \{ \begin{array}{rl}
		q_5\\
		-q_5
		\end{array}
		\right.$
\end{itemize}
Se $\beta\neq 0,\pi$ (non sono dunque in singolarità):
\begin{itemize}	

\item $\left \{ \begin{array}{rl}
		s(\beta)c(\gamma)=r_{13} \\
		s(\beta)s(\gamma)=r_{23}
		\end{array}
		\right. %il punto serve per non far chiamare errore
		\longrightarrow
		\left \{ \begin{array}{rl}
		c(\gamma)=\dfrac{r_{13}}{s(\beta)} \\
		s(\gamma)=\dfrac{r_{23}}{s(\beta)}
		\end{array}
		\right.$
			
$\gamma = atan2 (\pm s(\gamma), \pm c(\gamma)) =   atan2 (\pm \dfrac {r_{23}}{s(\beta)}\pm \dfrac {r_{13}}{s(\beta)})=
		\left \{ \begin{array}{rl}
		q_4\\
		q_4+\pi
		\end{array}
		\right.$
		
		
\item $\left \{ \begin{array}{rl}
		-s(\beta)c(\alpha)=r_{31} \\
		s(\beta)s(\alpha)=r_{32}
		\end{array}
		\right. %il punto serve per non far chiamare errore
		\longrightarrow
		\left \{ \begin{array}{rl}
		c(\alpha)=\dfrac{r_{31}}{-s(\beta)} \\
		s(\alpha)=\dfrac{r_{32}}{s(\beta)}
		\end{array}
		\right.$
				
		$\alpha = atan2 (\pm s(\alpha),\mp c(\alpha)) = atan2 (\pm \dfrac {r_{32}}{s(\beta)}\mp \dfrac {r_{31}}{s(\beta)})=
		\left \{ \begin{array}{rl}
		q_6\\
		q_6+\pi
		\end{array}
		\right.$

\end{itemize}

N.B. $s(\beta)$ che è al denominatore non è pari a  $\pm \sqrt{1-r_{33}^2}$ bensì al seno del valore di $\beta$ che ho ricavato.

\newpage
\section{Cinematica inversa completa: Manipolatore di Stanford + Polso Sferico}
\subsection{Calcolo posizione disaccoppiata}

Avendo già calcolato le due costanti che permettono il disaccoppiamento polso-struttura portante possiamo dire subito che: dati

\begin{center}
$d_f=\begin{pmatrix}{{\mathit{df}}_1}\\
{{\mathit{df}}_2}\\
{{\mathit{df}}_3}\end{pmatrix}$ e $R_f=\begin{pmatrix}{r_{11}} & {r_{12}} & {r_{13}}\\
{r_{21}} & {r_{22}} & {r_{23}}\\
{r_{31}} & {r_{32}} & {r_{33}}\end{pmatrix}$
\end{center}

\begin{flushleft}
dove $d_f$ è il vettore posizione del punto terminale del robot completo di Stanford (manipolatore + polso sferico), ricavato con la cinematica diretta del robot, rispetto al sistema di riferimento inerziale e $R_f$ è la matrice di rotazione che indica l'orientamento del punto terminale del robot completo di Stanford rispetto al sistema di riferimento inerziale. \\
L'equazione per ottenere le variabili $q_a$ che servono per posizionare il punto terminale del manipolatore di Stanford (cioè il polso sferico, infatti con $d_f- R_f \cdot d_1$ mi trovo sul polso) rispetto al sistema di riferimento di base è:
\end{flushleft}

\begin{center}
$d_f- R_f \cdot d_1 = R_{03}(q_a) \cdot d_0 + d_{03}(q_a)$ 
\end{center}

Svolgendo i calcoli:

\begin{center}
$\begin{pmatrix}{{\mathit{df}}_1}-{L_6} {r_{13}}\\
{{\mathit{df}}_2}-{L_6} {r_{23}}\\
{{\mathit{df}}_3}-{L_6} {r_{33}}\end{pmatrix}=\begin{pmatrix}-{L_2} {s_{{q_1}}}+{c_{{q_1}}} {s_{{q_2}}} {q_3}\\
{L_2} {c_{{q_1}}}+{s_{{q_1}}} {s_{{q_2}}} {q_3}\\
{L_1}+{c_{{q_2}}} {q_3}\end{pmatrix}$
\end{center}

Applicando la cinematica inversa del Manipolatore di Stanford ricavo il valore delle variabili di giunto che servono per posizionare il polso sferico rispetto al sistema di riferimento di base:
\begin{itemize}
%\item $q_3 = \sqrt{(df_1-L_6 r_{13})^2 +(df_2-L_6 r_{23})^2+(df_3-L_6 r_{33}-L_1)-L_2^2}$
\item $q_3 = \sqrt{x^2 +y^2+(z-L_1)^2-L_2^2}$
\item $q_2 = atan2(\pm\sqrt{\dfrac{x^2+y^2-L2^2}{q_3^2}},\dfrac{z-L_1}{q_3})$
\item $q_1 = atan2(q_3 s_{q_2}y-L_2x,q_3s_{q2}x+L_2y)$
\end{itemize}

\subsection{Calcolo Orientamento disaccoppiato}
Arrivati a questo punto basta calcolare il risultato di $ R_{03}^T(q_a) \cdot R_f $ per ottenere la matrice di orientamento del polso sferico $R_{36}(q_b)$. Calcolata la matrice $R_{36}(q_b)$ è possibile risalire (grazie alla cinematica inversa fatta nel capitolo 2: Orientamento inverso del Polso Sferico) ai valori degli ultimi tre giunti $q_4,q_5,q_6$ necessari per determinare l'orientamento del polso sferico.

\subsection{Conclusione}
Attraverso la conoscenza di $q_a$ e $q_b$ sono in grado di determinare l'orientamento e la posizione del punto terminale del robot:\\

		$\left\{ \begin{array} {ll} %ll (doppia elle) mi permette di inserire i due elementi presenti all'interno 			della parentesi graffa entrambi a sinistra
		R_f  = R_{03}(q_a) \cdot  R_{36}(q_b) \\
		d_f = R_{03}(q_a) \cdot d_{36}(q_b) + d_{03}(q_a)
		\end{array}
		\right.$

\newpage

\section{Cinematica inversa completa: Antropomorfo + Polso Sferico (Puma)}
\subsection{Calcolo posizione disaccoppiata}
Avendo già calcolato le due costanti che permettono il disaccoppiamento polso-struttura portante possiamo dire subito che: dati

\begin{center}
$d_f=\begin{pmatrix}{{\mathit{df}}_1}\\
{{\mathit{df}}_2}\\
{{\mathit{df}}_3}\end{pmatrix}$ e $R_f=\begin{pmatrix}{r_{11}} & {r_{12}} & {r_{13}}\\
{r_{21}} & {r_{22}} & {r_{23}}\\
{r_{31}} & {r_{32}} & {r_{33}}\end{pmatrix}$
\end{center}

\begin{flushleft}
dove $d_f$ è il vettore posizione del punto terminale del robot completo di Stanford (manipolatore + polso sferico), ricavato con la cinematica diretta del robot, rispetto al sistema di riferimento inerziale e $R_f$ è la matrice di rotazione che indica l'orientamento del punto terminale del robot completo di Stanford rispetto al sistema di riferimento inerziale. \\
L'equazione per ottenere le variabili $q_a$ che servono per posizionare il punto terminale dell'antropomorfo (cioè il polso sferico, infatti con $d_f- R_f \cdot d_1$ mi trovo sul polso) rispetto al sistema di riferimento di base è:
\end{flushleft}

\begin{center}
$d_f- R_f \cdot d_1 = R_{03}(q_a) \cdot d_0 + d_{03}(q_a)$ 
\end{center}

Svolgendo i calcoli:

\begin{center}
$\begin{pmatrix}{{\mathit{df}}_1}-{L_6} {r_{13}}\\
{{\mathit{df}}_2}-{L_6} {r_{23}}\\
{{\mathit{df}}_3}-{L_6} {r_{33}}\end{pmatrix}=\begin{pmatrix}{D_2} {c_{{q_1}}} {c_{{q_2}}}+{D_3} {c_{{q_1}}} {c_{{q_2}}} {c_{{q_3}}}+-{D_3} {c_{{q_1}}} {s_{{q_2}}} {s_{{q_3}}}\\
{D_2} {s_{{q_1}}} {c_{{q_2}}}+{D_3} {s_{{q_1}}} {c_{{q_2}}} {c_{{q_3}}}+-{D_3} {s_{{q_1}}} {s_{{q_2}}} {s_{{q_3}}}\\
{L_1}+{D_2} {s_{{q_2}}}+{D_3} {s_{{q_2}}} {c_{{q_3}}}+{D_3} {c_{{q_2}}} {s_{{q_3}}}\end{pmatrix}$
\end{center}

Applicando la cinematica inversa dell'Antropomorfo ricavo il valore delle variabili di giunto che servono per posizionare il polso sferico rispetto al sistema di riferimento di base:
\begin{itemize}
%\item $q_3 = \sqrt{(df_1-L_6 r_{13})^2 +(df_2-L_6 r_{23})^2+(df_3-L_6 r_{33}-L_1)-L_2^2}$
\item $q_3 = atan2(\pm\sqrt{1-a^2},a) \leftarrow a=\dfrac{x^2+y^2+(z-L_1)^2-D_3^2-D_2^2}{2D_2D_3}$
\item $q_2 = atan2(zD_3c_{q_3} + zD_2 - L_1D_3c_{q_3} - L_1D_2 \mp \sqrt{x^2+y^2}D_3s_{q_3},$

\hspace{20mm} $ \pm \sqrt{x^2+y^2}D_3c_{q_3} \pm \sqrt{x^2+y^2}D_2 + zD_3s_{q_3} - L_1D_3s_{q_3})$
\item $q_1 = atan2(\dfrac{y}{D_2c_{q_2}+D_3c_{q_2+q_3}},\dfrac{x}{D_2c_{q_2}+D_3c_{q_2+q_3}})$
\end{itemize}

\subsection{Calcolo Orientamento disaccoppiato}
Arrivati a questo punto basta calcolare il risultato di $ R_{03}^T(q_a) \cdot R_f $ per ottenere la matrice di orientamento del polso sferico $R_{36}(q_b)$. Calcolata la matrice $R_{36}(q_b)$ è possibile risalire (grazie alla cinematica inversa fatta nel capitolo 2: Orientamento inverso del Polso Sferico) ai valori degli ultimi tre giunti $q_4,q_5,q_6$ necessari per determinare l'orientamento del polso sferico.

\subsection{Conclusione}
Attraverso la conoscenza di $q_a$ e $q_b$ sono in grado di determinare l'orientamento e la posizione del punto terminale del robot:\\

		$\left\{ \begin{array} {ll} %ll (doppia elle) mi permette di inserire i due elementi presenti all'interno 			della parentesi graffa entrambi a sinistra
		R_f  = R_{03}(q_a) \cdot  R_{36}(q_b) \\
		d_f = R_{03}(q_a) \cdot d_{36}(q_b) + d_{03}(q_a)
		\end{array}
		\right.$



















\end{document}